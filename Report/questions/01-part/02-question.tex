\pagebreak
\question{[5]}

Describe your choice of distance function in SE(3) and why do you think that it
is appropriate for finding good neighborhoods.

The choice for total distance uses euclidean and orientation weights $w_e$ and
$w_o$ as well as euclidean and orientation distances $d_e$ and $d_o$ as
follows:
$$d = w_ed_e + w_od_o$$
subject to:
\begin{align*}
w_e + w_o &= 1 \\
d_e, d_o &\in [0,1]
\end{align*}
With the above restraints, we have $d \in [0,1]$.

For euclidean distance we scale the euclidean norm to the range $[0,1]$ as
follows:
$$d_e = \dfrac{\sqrt{x^2 + y^2 + z^2}}{d_{max}}$$
where we take $d_{max}$ to be the maximum possible euclidean distance in the
configuration space. The limited range on $d_e$ is used specifically for
allowing user control on the weights for euclidean versus orientation distances.

The choice of distance $d_o$ for quaternions $Q_1$ and $Q_2$ is the dot product
manipulated as follows:
$$d_o = 1 - |Q_1 \cdot Q_2|$$
Use of the dot product here is simply because it produces an approximation
to the length of the great arc circle over a foud dimensional sphere. This
can be seen by taking $cos^{-1}(Q_1 \cdot Q_2)$. There is no need for the actual
computation of the arccos since it is proportional the angle between two
vectors is proportional to the length of the arc at their tips.

Since we are dealing with unit quaternion, we know that $Q_1 \cdot Q_2$ will
fall in the range $[-1, 1]$. All of the rotations in the range $[-1,0)$ are mirror
images of the rotations in the range $(0,1]$, so the absolute value of the
dot product is taken to eliminate the bottom range.

This particular method is appropriate for finding good neighborhoods because it
focuses on approximating the distance following the actual path through our
configuration space in both the translation and orientation phases.
In terms of euclidean distance, the sortest path is the straight line distance
between two points, so this is the metric we wish to use.
In terms of orientation distance, it approximates the orientation path along
the great circle arc in orientation space, which is the shortest path between
two points on a hypersphere. Since there is no shorter distance, this is
precisely the path we wish to travel so this is precisely the distance we wish
to measure.