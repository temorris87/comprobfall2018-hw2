\question{[5]}

Describe your implementation for the local planner in SE(3), as well as how you
collision check the corresponding path, and why these are the right choices
for the problem.

The implementation for the local planner in SE(3) uses interpolation seperately
for euclidean space and orientation space. In euclidean space the interpolation
is done by dividing the line segment evenly into $n$ distinct points. For each
of those $n$ points, a point is found that corresponds at that specific point
in the orientation space. The orientation space is divided into $n$ points as
well using the spherical linear interpolation algorithm ($slerp$) on page 5 in the
Effective Sampling and Distance Metrics for Rigid Body Path Planning written
by James J. Kuffner. Combining each of the euclidean points with its
corresponding orientation, a conversion is made into a matrix that the PQP
algorithm accepts and testing for collision is performed.

These are the right choices because both the euclidean space and orientation
space between two points is evenly divided up and matched together. The $slerp$
algorithm is best for orientation because it follows the great arc of the
fourth dimensional hypersphere, giving the most accurate representation.
This will allow for the smoothest transition between translation and orientation
that is possible between these two points. The only difficulty arises in making
sure that $n$ is large enough that all collisions are detected on the path
between two points. If $n$ is too small, it is possible that the piano will jump
too far and miss noticing that a collision would have occured.