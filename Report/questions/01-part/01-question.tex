\question{[5]}

Describe your choice for representing configurations in SE(3) and how you are
sampling collision-free configurations. Argue that your sampling process
is indeed generating uniform at random configurations in the free
part of the configuration space.

The choice made for representing configurations in SE(3) is to use
$\mathbb{R}^3$ for euclidean distance and quaternions for orientation.
In order to determine if a given configuration is collision free we transform
it into a matrix that the PQP library accepts and use PQP to check for the
collision. Uniform sampline in $\mathbb{R}^3$ is simply done by taking random
values within the bounds of our configuration space for $(x, y, z)$.
For sampling SO(3) uniformly, the following algorithm is used:

\begin{algorithm}[H]
	\KwData {None}
	\KwResult {Random uniformly distributed uniform quaternion.}
	$s \gets rand()$\;
	$\sigma_1 \gets \sqrt{1-s}$\;
	$\sigma_2 \gets \sqrt{s}$\;
	$\theta_1 \gets 2\pi \cdot rand()$\;
	$\theta_2 \gets 2\pi \cdot rand()$\;
	$w \gets cos(\theta_2) \cdot \sigma_2$\;
	$x \gets sin(\theta_1) \cdot \sigma_1$\;
	$y \gets cos(\theta_1) \cdot \sigma_1$\;
	$z \gets sin(\theta_2) \cdot \sigma_2$\;
	$(w,x,y,z)$
\end{algorithm}

First we start by showing that this algorithm produces a unit length vector.
The quaternion $(w, x, y, z)$ is just a vector, and it's magnitude is:

\begin{align*}
  = &\sqrt{w^2 + x^2 + y^2 + z^2} \\
  = &\sqrt{cos^2(\theta_2)\cdot\sigma_2^2 + sin^2(\theta_1)\cdot\sigma_1^2 + cos^2(\theta_1)\cdot\sigma_1^2 + sin^2(\theta_2)\cdot\sigma_2^2} \\
  = &\sqrt{\sigma_2^2(cos^2(\theta_2) + sin^2(\theta_2)) + \sigma_1^2(cos^2(\theta_1) + sin^2(\theta_1))} \\
  = &\sqrt{\sigma_2^2 + \sigma_1^2} \\
  = &\sqrt{(\sqrt{1-s})^2 + (\sqrt{s})^2} \\
  = &\sqrt{1 - s + s} \\
  = &\sqrt{1} \\
  = &1
\end{align*}

So now that we know that the quaternions being produced are unit quaternions,
we now have to show that there is no possibility of two quaternions pointing
to the same position on a unit sphere. To do this we first take note that
$\theta_1$ and $\theta_2$ are coordinates on a torus. Given two coordinates in
this manner, we can find a \emph{unique} point along the surface of a torus.
Now we have to look at what effect $s$ has on our computation. What $s$ does
is essentially cut out a particular torus from four dimensional space. We can
visualize this if we look at $\sigma_1$ as being the radius from the hole
of the torus to the center of an edge and view $\sigma_2$ as the radius of a
cross section of the edge.

From the above argument we conclude that this algorithm first picks a
particular torus from four dimensional space, then uniquely identifies
points along the surface of that torus. Since there is no overlap we conclude
that this process generates uniform at random configurations for the
configuration SO(3). Since random values $(x,y,z)$ will be uniform in
$\mathbb{R}^3$, we further conclude that this process generates uniform at
random configurations for SE(3).
